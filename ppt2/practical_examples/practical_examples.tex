\documentclass[../main/main.tex]{subfiles}

\begin{document}
\section{Adaptatividade de malha}


%--------------------------------- 
\setLayout{mainpoint}
\begin{frame}
  \frametitle{Adaptatividade de malha}
\end{frame}


%--------------------------------- 
\setLayout{vertical}

\begin{frame}{Introdução às Malhas Dinâmicas}
  \begin{columns}
    \begin{column}{0.6\textwidth}
      \begin{itemize}
        \item \textbf{Definição}: Malhas que se adaptam a mudanças na geometria durante simulações transientes
        \item \textbf{Objetivo}: Manter qualidade e validade da malha sem intervenção do usuário
        \item \textbf{Métodos Principais}:
          \begin{itemize}
            \item \textbf{Método R}: Rearranjo de nós
            \item \textbf{Método H}: Refinamento/desrefinamento
            \item \textbf{Método P}: Enriquecimento polinomial
          \end{itemize}
        \item \textbf{Aplicações}: Escoamentos com superfícies móveis, FSI, turbomáquinas
      \end{itemize}
    \end{column}
    \begin{column}{0.4\textwidth}
      \begin{figure}
        \centering
        \includegraphics[width=\textwidth]{imgs/tipos_refinamento_adaptativo.png}
        \caption{Tipos de refinamento adaptativo}
      \end{figure}
    \end{column}
  \end{columns}
\end{frame}

\begin{frame}{Aplicações e Casos de Estudo}
  \begin{itemize}
    \item \textbf{Cilindro em Canal}:
      \begin{itemize}
        \item Comparação Laplaciano vs Pseudo-Sólido
        \item Deslocamento máximo sem inversão
      \end{itemize}

    \item \textbf{Turbina Eólica (VAWT)}:
      \begin{itemize}
        \item Malhas deslizantes para rotação
        \item Domínio interno rotativo e externo fixo
      \end{itemize}

    \item \textbf{Bomba Micro}:
      \begin{itemize}
        \item Método ALE para FSI
        \item Cantilevers como válvulas
      \end{itemize}

    \item \textbf{Refinamento Adaptativo Dinâmico (AMR)}:
      \begin{itemize}
        \item Captura de gradientes elevados
        \item Aplicação em indutor espiral
      \end{itemize}
    \item Vídeo de exemplo: \url{https://youtu.be/AiDFNevhJ98?si=Do-D2FmBH5UynQNJ}
  \end{itemize}
\end{frame}

\end{document}
