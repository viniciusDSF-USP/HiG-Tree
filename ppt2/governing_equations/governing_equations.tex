\documentclass[../main/main.tex]{subfiles}

\begin{document}
\section{Equações governantes}

%--------------------------------- 
\setLayout{mainpoint}
\begin{frame}
  \frametitle{Equações governantes}
\end{frame}

% %--------------------------------- 
% \setLayout{vertical}
% \begin{frame}
%   Teste
% \end{frame}

\setLayout{vertical}
\begin{frame}{Hipóteses}
  \begin{block}{Características do Escoamento}
    \begin{itemize}
      \item \textbf{Não estacionário}
      \item \textbf{Escoamento laminar}
      \item \textbf{Escoamento isotérmico}
      \item \textbf{Fluidos incompressíveis}
      \item \textbf{Duas fases imiscíveis}
      \item \textbf{Sem transferência de massa através da interface}
    \end{itemize}
  \end{block}
\end{frame}

\setLayout{vertical}
\begin{frame}{Conservação da Massa}
  \begin{block}{Equação da Continuidade}
    \[
      \nabla \cdot \mathbf{u} = 0
    \]
  \end{block}
  \vspace{0.5cm}
  \begin{itemize}
    \item $\mathbf{u} = (u, v, w)$: campo de velocidade
    \item Condição de incompressibilidade
    \item Descreve a conservação da massa para fluidos incompressíveis
  \end{itemize}
\end{frame}

\setLayout{vertical}
\begin{frame}{Balanço de Momentum Linear}
  \begin{block}{Forma Dimensional}
    \scriptsize
    \[
      \rho\left[\frac{\partial\mathbf{u}}{\partial t} + \mathbf{u}(\nabla\cdot\mathbf{u})\right] = -\nabla p + \nabla\cdot\left\{\mu\left[\nabla\mathbf{u} + (\nabla\mathbf{u})^{T}\right]\right\} + \rho\mathbf{g} + \sigma\kappa\mathbf{n}\delta
    \]
  \end{block}
  \vspace{0.3cm}
  \begin{columns}[T]
    \begin{column}{0.5\textwidth}
      \begin{itemize}
        \item $\rho$: densidade
        \item $p$: pressão
        \item $\mu$: viscosidade absoluta
        \item $\mathbf{g}$: campo gravitacional
      \end{itemize}
    \end{column}
    \begin{column}{0.5\textwidth}
      \begin{itemize}
        \item $\sigma$: coeficiente de tensão interfacial
        \item $\kappa$: curvatura da interface
        \item $\mathbf{n}$: vetor normal à interface
        \item $\delta$: função delta (1 na interface, 0 fora)
      \end{itemize}
    \end{column}
  \end{columns}
\end{frame}

\begin{frame}{Forma Adimensional}
  \begin{block}{Equação do Momentum Adimensional}
    \scriptsize
    \[
      \rho\left[\frac{\partial\mathbf{u}}{\partial t} + \mathbf{u}(\nabla\cdot\mathbf{u})\right] = -\nabla p + \frac{1}{Re}\nabla\cdot\left\{\mu\left[\nabla\mathbf{u} + (\nabla\mathbf{u})^{T}\right]\right\} + \rho\frac{\mathbf{g}}{g} + \frac{1}{Bo}\kappa\mathbf{n}\delta
    \]
  \end{block}
  \vspace{0.5cm}
  \begin{itemize}
    \item $Re = \dfrac{\rho_1 g^{1/2} L^{3/2}}{\mu_1}$: Número de Reynolds
    \item $Bo = \dfrac{\rho_1 g L^2}{\sigma}$: Número de Bond (Eötvös)
    \item $L$: comprimento característico
    \item $V = \sqrt{g L}$: velocidade característica
  \end{itemize}
\end{frame}

\begin{frame}{Transporte da Fase - Método VOF}
  \begin{block}{Equação de Transporte da Fração Volumétrica}
    \[
      \frac{\partial f}{\partial t} + \nabla \cdot (\mathbf{u} f) = 0
    \]
  \end{block}
  \vspace{0.5cm}
  \begin{itemize}
    \item $f$: fração volumétrica ($0 \leq f \leq 1$)
    \item Reconstrução geométrica da interface via método PLIC
    \item Algoritmo de advecção com passo fracionado (split)
    \begin{itemize}
      \item Evita transporte duplo do mesmo fluido
    \end{itemize}
  \end{itemize}
\end{frame}

% \begin{frame}{Reconstrução da Interface - Método PLIC}
%   \begin{block}{Plano de Reconstrução}
%     \[
%       \mathbf{n} \cdot \mathbf{x} = \alpha
%     \]
%   \end{block}
%   \vspace{0.5cm}
%   \begin{itemize}
%     \item $\mathbf{n} = \nabla f / ||\nabla f||$: vetor normal à interface
%     \item $\mathbf{x}$: ponto no plano de reconstrução
%     \item $\alpha$: menor distância do plano à origem da célula
%     \item Volume abaixo do plano: $\bar{f}(\alpha, \mathbf{n}, \Delta) = f$
%     \item Cálculo de $\alpha$ resolve equação não-linear
%   \end{itemize}
% \end{frame}
%
% \begin{frame}{Advecção da Interface com Correção}
%   \begin{block}{Equação Modificada}
%     \[
%       \frac{\partial f}{\partial t} + \nabla \cdot (\mathbf{u} f) = f \nabla \cdot \mathbf{u}
%     \]
%   \end{block}
%   \vspace{0.3cm}
%   \begin{itemize}
%     \item Correção de divergência para conservação precisa
%     \item Satisfaz restrições locais e globais de volume
%     \item Transporte fracionado em três direções:
%       \begin{itemize}
%         \item $f^{n} \to f^{*}$ (direção $x$)
%         \item $f^{*} \to f^{**}$ (direção $y$)
%         \item $f^{**} \to f^{n+1}$ (direção $z$)
%       \end{itemize}
%   \end{itemize}
% \end{frame}
%
% \begin{frame}{Força de Tensão Superficial - Modelo CSF}
%   \begin{block}{Força de Superfície Contínua}
%     \[
%       \mathbf{F}_\sigma = \sigma \kappa \nabla f
%     \]
%   \end{block}
%   \vspace{0.5cm}
%   \begin{itemize}
%     \item $\sigma$: coeficiente de tensão interfacial
%     \item $\kappa$: curvatura da interface
%     \item $\nabla f$: gradiente da fração volumétrica
%     \item Modela a força capilar na interface
%   \end{itemize}
% \end{frame}
%
% \begin{frame}{Cálculo da Curvatura - Método da Função Altura}
%   \begin{itemize}
%     \item Stencil $7 \times 3 \times 3$ (ou $3 \times 7 \times 3$, $3 \times 3 \times 7$)
%     \item Funções altura na direção normal predominante:
%       \[
%         h_{i,j} = \sum_{k=-3}^{3} f_{i,j,k} \Delta z
%       \]
%     \item Cálculo da curvatura:
%       \scriptsize
%       \[
%         \kappa = \frac{h_{xx} + h_{yy} + h_{xx}h_y^2 + h_{yy}h_x^2 - 2h_{xy}h_x h_y}{(1 + h_x^2 + h_y^2)^{3/2}}
%       \]
%       \normalsize
%     \item Derivadas calculadas por diferenças finitas centradas
%   \end{itemize}
% \end{frame}
%
% \begin{frame}{Restrições do Passo Temporal}
%   \begin{itemize}
%     \item \textbf{Advecção:}
%       \[
%         \Delta t < \frac{\Delta}{2|\mathbf{u}|} \equiv \Delta t_{\text{Advecção}}
%       \]
%     \item \textbf{Tensão superficial:}
%       \[
%         \Delta t < \sqrt{\frac{(\rho_1 + \rho_2)\Delta^3}{4\pi\sigma}} \equiv \Delta t_{\sigma}
%       \]
%     \item \textbf{Termo parabólico (viscosidade):}
%       \[
%         \Delta t < Re \cdot \Delta^2 \equiv \Delta t_{\text{Parabólico}}
%       \]
%     \item \textbf{Termo hiperbólico (inércia):}
%       \[
%         \Delta t < \frac{\Delta}{\rho |\mathbf{u}|} \equiv \Delta t_{\text{Hiperbólico}}
%       \]
%     \item \textbf{Passo final:}
%       \[
%         \Delta t = \min(\Delta t_{\text{Advecção}}, \Delta t_{\sigma}, \Delta t_{\text{Parabólico}}, \Delta t_{\text{Hiperbólico}})
%       \]
%   \end{itemize}
% \end{frame}


\end{document}
