\documentclass[../main/main.tex]{subfiles}

\begin{document}
\section{Validação}


%---------------------------------     
\setLayout{mainpoint}
\begin{frame}
    \frametitle{Validação}
\end{frame}

% seção 4.0 
%--------------------------------- 
\setLayout{vertical}
\begin{frame}
    \begin{block}{}
        \smaller
        Os testes de validação empregam um campo de velocidade prescrito, de modo a isolar o desempenho do algoritmo de advecção dos efeitos da dissipação numérica do resolvedor de fluxo. A principal métrica quantitativa considerada é o Erro Absoluto de Volume ($\epsilon_{L_{1}}$), o qual quantifica a conservação de massa por meio da diferença absoluta entre os volumes inicial e final ($|V_{I} - V_{F}|$). Assim:
        \begin{equation*}
            \epsilon_{L_{1}} = |V_{I} - V_{F}|
        \end{equation*}
    \end{block}
\end{frame}





% seção 4.1 3D Shearing Flow Test
%--------------------------------- 
\setLayout{vertical}
\begin{frame}
    \frametitle{\smaller Teste de Escoamento por Cisalhamento 3D}
    
    \begin{block}{\smaller Configuração do Domínio}
        \begin{itemize}
            \smaller
            \item  Domínio cúbico tridimensional, $[0,1]\times[0,1]\times[0,2]$;
            \item Bolha inicial esférica, raio $R = 0.15$ com centro em $(0.5, 0.75, 0.25)$;
            \item  Campo de velocidade: 
            \begin{equation}
                \vec{u}=\begin{cases}
                                    u=-sin(\pi x)^{2}sin(2\pi y)cos(\frac{\pi t}{T}) \\ 
                                    v=sin(\pi y)^{2}sin(2\pi x)cos(\frac{\pi t}{T})\\  
                                    w=u_{max}(1-\frac{r}{R})^{2}cos(\frac{\pi t}{T})
                               \end{cases}
                 \label{eq_u1}
            \end{equation} 
            \item Tempo total $T = 6.0$
            \item  Passo de tempo de $\Delta t=1\times10^{-3}$ 
        \end{itemize}
    \end{block}
    
 \end{frame}
 
 
 \begin{frame}
     \begin{figure}[H]
         \includegraphics[width=0.6\textwidth]{imgs/Campo_1.png}
         \caption{\smaller Gráfico do campo vetorial da Equação (\ref{eq_u1}), instante $t = 0$}
     \end{figure}
 \end{frame}
 
 
 
   \begin{frame}
     \begin{figure}
         \includegraphics[width=0.6\textwidth]{imgs/fig-3.png}
         \caption{\smaller Comparação do formato da interface em $t = 3$ para o teste de escoamento cisalhante 3D. Resultados obtidos com (a) HiG-Flow e (b) o método PCFSC de \cite{ref14}, ambos utilizando uma malha de $64 \times 64 \times 128$ elementos.
         }
         \footnotesize Fonte: \cite{higflow}
     \end{figure}
 \end{frame}

   
  \begin{frame}
      \begin{table}[h!]
          \centering
          \begin{tabular}{lccc}
              \toprule
              \textbf{Solver} & \textbf{Mesh} & \textbf{Error ($\epsilon_{L_{1}}$)} \\
              \midrule
              HiG-Flow & $50 \times 50 \times 100$ & $2.86 \times 10^{-4}$ \\
              HiG-Flow & $64 \times 64 \times 128$ & $4.48 \times 10^{-5}$ \\
              CVTNA + unsplit & $64 \times 64 \times 128$ & $3.64 \times 10^{-3}$ \\
              ELVIRA + COSMIC & $64 \times 64 \times 128$ & $3.97 \times 10^{-3}$ \\
              CLC-CBIR & $64 \times 64 \times 128$ & $3.27 \times 10^{-3}$ \\ \bottomrule
              \hline
          \end{tabular}
          \caption{Comparação dos erros $L_{1}$ na conservação da fração de volume no teste de escoamento cisalhante 3D em $t = 6$, incluindo os resultados do HiG-Flow e os reportados por \cite{ref14}, \cite{ref39}, \cite{ref40}}
          \footnotesize Fonte: \cite{higflow}
      \end{table}
  \end{frame}
  

  
  

% seção 4.2 Teste de Campo de Deformação 3D
%--------------------------------- 
\setLayout{vertical}
\begin{frame}
    \frametitle{Teste de Campo de Deformação 3D}
    
     \begin{block}{\smaller Configuração do Domínio}
        \begin{itemize}
            \smaller
            \item  Domínio cúbico tridimensional, $[0,1]\times[0,1]\times[0,1]$;
            \item Bolha inicial esférica, raio $R = 0.15$ com centro em $(0.35, 0.35, 0.35)$;
            \item  Campo de velocidade: 
            \begin{equation}
                \vec{u} = \begin{cases}
                    u = 2\sin(\pi x)^2 \sin(2\pi y) \sin(2\pi z) \cos(\pi t/T) \\
                    v = -\sin(\pi y)^2 \sin(2\pi x) \sin(2\pi z) \cos(\pi t/T)  \\
                    w = \sin(\pi z)^2 \sin(2\pi x) \sin(2\pi y) \cos(\pi t/T) 
                \end{cases}
                \label{eq_u2}
            \end{equation} 
            \item Tempo total $T = 3.0$
            \item  Passo de tempo de $\Delta t=1\times10^{-3}$ 
        \end{itemize}
    \end{block}
    
\end{frame}


 \begin{frame}
    \begin{figure}[H]
        \includegraphics[width=0.6\textwidth]{imgs/campo_2.png}
        \caption{\smaller Gráfico do campo vetorial da Equação (\ref{eq_u2}), instante $t = 0$}
    \end{figure}
\end{frame}


\begin{frame}
    \begin{figure}
        \includegraphics[width=0.6\textwidth]{imgs/fig-5.png}
        \caption{\smaller Comparação do formato da interface em $t = 3$ para o teste de campo de deformação 3D. Resultados obtidos com (a) HiG-Flow e (b) o método PCFSC de \cite{ref14}, ambos utilizando uma malha com $128^{3}$ elementos.}
        \footnotesize Fonte: \cite{higflow}
    \end{figure}
\end{frame}




\begin{frame}
    \begin{table}[h!]
        \centering
        \begin{tabular}{lccc}
            \toprule
            \textbf{Solver} & \textbf{Mesh} & \textbf{Error ($\epsilon_{L_{1}}$)} \\
            \midrule
            HiG-Flow & $50 \times 50 \times 50$ & $1.15 \times 10^{-4}$ \\
            HiG-Flow & $64 \times 64 \times 64$ & $2.19 \times 10^{-5}$ \\
            HiG-Flow & $128 \times 128 \times 128$ & $9.41 \times 10^{-6}$ \\
            CVTNA + PCFSC unsplit \cite{ref14} & $64 \times 64 \times 64$ & $1.99 \times 10^{-3}$ \\
            CVTNA + PCFSC unsplit \cite{ref14} & $128 \times 128 \times 128$ & $3.09 \times 10^{-4}$ \\
            CLC-CBIR \cite{ref40} & $64 \times 64 \times 64$ & $2.09 \times 10^{-3}$ \\
            CLC-CBIR \cite{ref40} & $128 \times 128 \times 128$ & $3.52 \times 10^{-4}$ \\ 
            \bottomrule
        \end{tabular}
        \caption{\smaller Comparação dos erros $L_{1}$ na conservação da fração de volume no teste de campo de deformação 3D, incluindo os resultados do HiG-Flow e os reportados por \cite{ref14} e \cite{ref40}.}
        \footnotesize Fonte: \cite{higflow}
    \end{table}    
\end{frame}



% seção 4.3. 3D Buoyancy-Driven Rising Drop
%--------------------------------- 
\setLayout{vertical}
\begin{frame}
    \frametitle{\smaller Bolha ascendente 3D por empuxo}
    
    \begin{block}{Configuração do domínio}
        \begin{itemize}
            \item Domínio cúbico tridimensional, $[0,1]\times[0,1]\times[0,2]$;
            \item bolha inicial esférica, raio $R = 0.25$ com centro em $(0.5,0.5,0.5)$;
            \item condições de contorno $\mathbf{u} = 0$ nas paredes;
            \item gravidade $g = 0.98$;
            \item malha de $50\times50\times100$ células; e
            \item intervalo de tempo $\Delta t = 2\times 10^{-3}$.
        \end{itemize}
    \end{block}
\end{frame}

\begin{frame}{\smaller Casos simulados}
    \begin{table}[h!]
        \centering
        \small
        \begin{tabular}{lcccccccc}
            \toprule
            Caso & $\rho_1$ & $\rho_2$ & $\mu_1$ & $\mu_2$ & $g$  & $\sigma$ & $\mathrm{Re}$ & $\mathrm{Bo}$ \\
            \midrule
            1    & 1000     & 100      & 10      & 1       & 0.98 & 24.5     & 35            & 10            \\
            2    & 1000     & 1        & 10      & 0.1     & 0.98 & 1.96     & 35            & 125           \\
            \bottomrule
        \end{tabular}
    \end{table}

    O fluido externo tem densidade $\rho_1$ e viscosidade $\mu_1$, enquanto o fluido interno tem densidade $\rho_2$ e viscosidade $\mu_2$.

    \begin{table}[h!]
        \centering
        \small
        \begin{tabular}{lccl}
            \toprule
            Caso & $\rho_1/\rho_2$ & $\mu_1/\mu_2$ &                      \\
            \midrule
            1    & 10              & 10            & bolha quase esférica \\
            2    & 1000            & 100           & bolha se achata      \\
            \bottomrule
        \end{tabular}
    \end{table}

    \begin{itemize}
        \item Número de Reynolds mede a razão entre as forças inerciais e forças viscosas.
        \item Número de Bond mede a razão entre empuxo e forças de tensão superficial.
    \end{itemize}
\end{frame}

\begin{frame}{Resultados, caso 1}
    \begin{figure}
        \includegraphics[width=0.7\textwidth]{imgs/fig-07_silva-et-al.png}
        \caption{Comparação do formato da bolha em $t = 3$ com HiG-Flow~(esq.) e OpenFOAM~(dir.).}
    \end{figure}
\end{frame}

\begin{frame}
    \begin{figure}
        \includegraphics[width=0.7\textwidth]{imgs/fig-09_silva-et-al.png}
        \caption{Evolução do centro de massa da bolha~(esq.) e velocidade de ascensão da bolha~(dir.).}
    \end{figure}

\end{frame}

\begin{frame}{Resultados, caso 2}
    \begin{figure}
        \includegraphics[width=0.7\textwidth]{imgs/fig-10_silva-et-al.png}
        \caption{Comparação do formato da bolha em $t = 3$ com HiG-Flow~(esq.) e OpenFOAM~(dir.).}
    \end{figure}
\end{frame}

\begin{frame}
    \begin{figure}
        \includegraphics[width=0.7\textwidth]{imgs/fig-11_silva-et-al.png}
        \caption{Comparação do formato da bolha em $t = 3$ com HiG-Flow~(esq.) e FEEL~(dir.).}
    \end{figure}
\end{frame}

\begin{frame}
    \begin{figure}
        \includegraphics[width=0.7\textwidth]{imgs/fig-12_silva-et-al.png}
        \caption{Evolução do centro de massa da bolha~(esq.) e velocidade de ascensão da bolha~(dir.).}
    \end{figure}
\end{frame}



\end{document}