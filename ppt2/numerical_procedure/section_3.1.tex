\documentclass[../main/main.tex]{subfiles}

\begin{document}
\subsection{HiG-Tree e HiG-Flow}

\begin{frame}{HiG-Tree e HiG-Flow Software - Estrutura de Dados}
  \begin{itemize}
    \item \textbf{HiG-Tree}: estrutura de dados hierárquica em malha cartesiana
    \item Elementos de tamanhos variáveis com refinamento espacial recursivo
    \item Baseada em \textbf{m-tree} (Exemplo na figura)
    \item Malha não-estruturada com refinamento arbitrário
    \item Projetada para ser genérica e extensível a qualquer número de dimensões
  \end{itemize}
  \begin{center}
    \includegraphics[width=0.8\textwidth]{imgs/hig-tree-structure.png}
  \end{center}
\end{frame}

\begin{frame}{HiG-Tree - Particionamento de Domínio}
  \begin{itemize}
    \item Domínios compostos por blocos para geometrias complexas
    \item Discretização usando estrutura HiG-Tree
    \item Particionamento com biblioteca \textbf{Zoltan-Trilinos}
      \begin{itemize}
        \item Garante boa distribuição de carga entre processos
      \end{itemize}
    \item Armazenamento e enumeração de propriedades em células e facetas
    \item Refinamento espacial apropriado para cada problema simulado
  \end{itemize}
\end{frame}

\begin{frame}{HiG-Tree - Aproximações e Interpolações}

  \begin{columns}
    
    % Coluna da esquerda - Texto
    \begin{column}{0.5\textwidth}
      \begin{itemize}
        \item Módulo para estimativa de propriedades em células e facetas
        \item Método dos \textbf{Mínimos Quadrados Móveis (MLS)}
        \item Seleção da ordem polinomial de aproximação
        \item Elimina dependência geométrica usando nuvem de pontos vizinhos
        \item Essencial para malhas com elementos de diferentes níveis de refinamento
      \end{itemize}
    \end{column}
    
    % Coluna da direita - Imagem
    \begin{column}{0.5\textwidth}
      \begin{center}
        \includegraphics[width=\textwidth]{imgs/stencil.png}
      \end{center}
    \end{column}
    
  \end{columns}

\end{frame}
\begin{frame}{Exemplo: Aproximação por Diferenças Finitas}
  \begin{block}{Derivada Segunda com Interpolação MLS}
    \[
      \frac{\partial^{2}u}{\partial x^{2}} \approx \frac{1}{\Delta x^{2}}(u_{l} - 2u_{c} + u_{r})
    \]
  \end{block}
  \begin{itemize}
    \item Ponto $u_{l}$ não coincide com a malha → requer interpolação
    \item Interpolação MLS:
      \[
        u_{l} = \sum_{k \in \mathcal{I}_{l}} w^{l}_{k} u_{k}
      \]
    \item $\mathcal{I}_{l}$: conjunto de índices dos vizinhos de $u_{c}$
    \item $w^{l}_{k}$: pesos calculados pelo método MLS
    \item $N_{l}$: número de vizinhos para manter a ordem de precisão
  \end{itemize}
\end{frame}

\begin{frame}{HiG-Tree - Solução de Sistemas Lineares}
  \begin{itemize}
    \item Duas bibliotecas poderosas para computação paralela:
      \begin{itemize}
        \item \textbf{HYPRE}: High Performance Preconditioners
        \item \textbf{PETSc}: Portable, Extensible Toolkit for Scientific Computation
      \end{itemize}
    \item Otimizadas para arquiteturas de memória compartilhada e distribuída
    \item Rotinas de alto desempenho testadas em diversos códigos de CFD
    \item Tolerância de convergência tipicamente $1 \times 10^{-7}$
  \end{itemize}
\end{frame}

\begin{frame}{HiG-Flow - Capacidades de Simulação}
  \begin{itemize}
    \item Simulação de escoamentos monofásicos:
      \begin{itemize}
        \item Fluidos Newtonianos
        \item Generalizados Newtonianos
        \item Viscoelásticos
      \end{itemize}
    \item Simulação de escoamentos bifásicos:
      \begin{itemize}
        \item Método VOF para representação de interface
      \end{itemize}
    \item Dimensão arbitrária (1D, 2D, 3D, ..., ND)
    \item Design modular para implementação fácil de novas técnicas
  \end{itemize}
\end{frame}

\begin{frame}{HiG-Flow - Esquemas Temporais}
  \begin{columns}[T]
    \begin{column}{0.5\textwidth}
      \begin{block}{Métodos Explícitos}
        \begin{itemize}
          \item Euler explícito
          \item Runge-Kutta TVD 2ª ordem (Euler modificado)
          \item Runge-Kutta TVD 3ª ordem
        \end{itemize}
      \end{block}
    \end{column}
    \begin{column}{0.5\textwidth}
      \begin{block}{Métodos Implícitos}
        \begin{itemize}
          \item Euler implícito
          \item Crank-Nicholson
          \item BDF 2ª ordem (Backward Differentiation Formula)
        \end{itemize}
      \end{block}
    \end{column}
  \end{columns}
  \vspace{0.5cm}
  \begin{itemize}
    \item Métodos de projeção para acoplamento pressão-velocidade:
      \begin{itemize}
        \item Incremental (2ª ordem no tempo)
        \item Não-incremental (1ª ordem no tempo)
      \end{itemize}
  \end{itemize}
\end{frame}



\end{document}
