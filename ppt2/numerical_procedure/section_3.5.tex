\documentclass[../main/main.tex]{subfiles}

\begin{document}
\subsection{Tensão superficial e Curvatura}

\begin{frame}{Força de Tensão Superficial - Modelo CSF}
  \begin{block}{Força de Superfície Contínua}
    \[
      \mathbf{F}_\sigma = \sigma \kappa \nabla f
    \]
  \end{block}
  \vspace{0.5cm}
  \begin{itemize}
    \item $\sigma$: coeficiente de tensão interfacial
    \item $\kappa$: curvatura da interface
    \item $\nabla f$: gradiente da fração volumétrica
    \item Modela a força capilar na interface
  \end{itemize}
\end{frame}

\begin{frame}{Cálculo da Curvatura - Método da Função Altura}
  \begin{itemize}
    \item Stencil $7 \times 3 \times 3$ (ou $3 \times 7 \times 3$, $3 \times 3 \times 7$)
    \item Funções altura na direção normal predominante:
      \[
        h_{i,j} = \sum_{k-3}^{k+3} f_{i,j,k} \Delta z
      \]
    \item Cálculo da curvatura:
      \scriptsize
      \[
        \kappa = \frac{h_{xx} + h_{yy} + h_{xx}h_y^2 + h_{yy}h_x^2 - 2h_{xy}h_x h_y}{(1 + h_x^2 + h_y^2)^{3/2}}
      \]
      \normalsize
    \item Derivadas calculadas por diferenças finitas centradas
  \end{itemize}
\end{frame}

\end{document}
