\documentclass[../main/main.tex]{subfiles}

%_______________________________________________________________________________________________________________________
\begin{document}



\section{Exemplo de Poisson}





%-------Slide 1-------	
%----------------------------------------------------------------------------------------------------------------------	
\setLayout{mainpoint}
\begin{frame}
	\frametitle{Exemplo de Poisson}
\end{frame}
%----------------------------------------------------------------------------------------------------------------------	



%-------Slide 2-------	
%----------------------------------------------------------------------------------------------------------------------	
\setLayout{vertical}
% \begin{frame}{Título}
% 	...
% \end{frame}
%----------------------------------------------------------------------------------------------------------------------	

\begin{frame}{Equação de Poisson}
	A equação de Poisson é uma EDP da forma\[
		\nabla^2 u(x) = f(x),\quad x\in\Omega,
	\] em que $\nabla^2 u = \operatorname{div}(\operatorname{grad}(u))$.
\end{frame}

\begin{frame}{Condições de fronteira}
	\begin{itemize}
		\item Dirichlet: $u|_{\partial\Omega} = g_D$.
		\item Neumann: $\partial u/\partial n = g_N$ em $\partial\Omega$.
		\item Robin: $\alpha u + \beta\partial u/\partial n = h$ em $\partial\Omega$.
	\end{itemize}
\end{frame}

\begin{frame}{Diferenças finitas}
	Considera-se $\Omega = [a,b]\times[c,d]$. Usando uma malha cartesiana regular com espaçamentos $\delta x$ e $\delta y$, obtém-se\[
		\frac{u_{i+1,j}-2u_{i,j}+u_{i-1,j}}{\delta x^{2}}+\frac{u_{i,j+1}-2u_{i,j}+u_{i,j-1}}{\delta y^{2}}+\mathcal{O}(\delta x^{2}+\delta y^{2})=f_{i,j},
	\]\pause
	ou ainda, desprezando-se os termos de ordem 2,
	\[
		\frac{U_{i+1,j}-2U_{i,j}+U_{i-1,j}}{\delta x^{2}}+\frac{U_{i,j+1}-2U_{i,j}+U_{i,j-1}}{\delta y^{2}}=f_{i,j},
	\]\pause
	onde $U_{i,j}$ representa uma aproximação para a função $u_{i,j}$, avaliada em $(x_{i},y_{j}) = (a+i\delta x, b+j\delta y)$, e $f_{i,j}=f(x_i, y_j)$.
\end{frame}

\end{document}
%_______________________________________________________________________________________________________________________
