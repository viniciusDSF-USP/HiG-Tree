\documentclass[../main/main.tex]{subfiles}

%_______________________________________________________________________________________________________________________
\begin{document}


\section{Mínimos quadrados}

\setLayout{mainpoint}
\begin{frame}
	\frametitle{Mínimos quadrados, mínimos quadrados ponderados, mínimos quadrados móveis}
\end{frame}
%----------------------------------------------------------------------------------------------------------------------	

\setLayout{vertical}
\begin{frame}{Mínimos quadrados -- LS}

	Seja $f$ uma função e $V$ um espaço vetorial $(n+1)$-dimensional gerado pelas funções $\varphi_0,\varphi_1,\dotsc,\varphi_n$. Deseja-se encontrar $F^*\in V$ com $F^* = \sum_{i=0}^n \alpha_i^*\varphi_i$ que melhor aproxima $f$, i.e., quer-se obter $Q = \min_{\alpha^*}\lVert f - F^*\rVert^2$.
\end{frame}

\begin{frame}
	Para encontrar $F^*$, basta calcular e resolver o sistema de equações normais\[
		\begin{bmatrix}
			\langle\varphi_0,\varphi_0\rangle & \cdots & \langle\varphi_0,\varphi_n\rangle \\
			\vdots                            & \ddots & \vdots                            \\
			\langle\varphi_n,\varphi_0\rangle & \cdots & \langle\varphi_n,\varphi_n\rangle
		\end{bmatrix}\begin{bmatrix}
			\alpha_0^* \\
			\vdots     \\
			\alpha_n^*
		\end{bmatrix} = \begin{bmatrix}
			\langle f,\varphi_0\rangle \\
			\vdots                     \\
			\langle f,\varphi_n\rangle
		\end{bmatrix}
	\]
\end{frame}

\begin{frame}{Mínimos quadrados ponderados -- WLS}
	Vê-se que ${\langle\cdot,\cdot\rangle}_W$ dado por ${\langle u,v\rangle}_W = \sum_{i=0}^n w_i u_i v_i = u^\intercal W v$ com $W = \operatorname{diag}(w_0,\dotsc,w_n)$ e $w_i > 0$ define um produto interno em $\mathbb{R}^{n+1}$.\pause

	O método dos mínimos quadrados ponderados generaliza LS ao considerar produtos internos da forma ${\langle\cdot,\cdot\rangle}_W$.\pause

	Nota-se que LS é WLS com $W = I$.
\end{frame}

\begin{frame}{Mínimos quadrados móveis -- MLS}
	Consideremos $f$ uma função discreta definida em $n+1$ pontos. Seja $V$ um espaço vetorial $(m+1)$-dimensional gerado por $\{\varphi_0,\dotsc,\varphi_m\}$ com $m < n$.\pause

	Quer-se aproximar $f$ por uma função $F^*\in V$ por mínimos quadrados ponderados.\pause

	Definamos $w$ uma função peso dada por\[
		w(x) = \psi\left(\frac{|x - c|}{\sigma}\right),\quad \psi(t) = e^{-t^2},
	\] onde $c$ é o centro da gaussiana com raio controlado por $\sigma$.
\end{frame}

\begin{frame}
	Definamos ${\langle\cdot,\cdot\rangle}_{W(x)}$ um produto interno dado por\[
		{\langle u,v\rangle}_{W(x)} = \sum_{i=0}^n w_i(x)u_i v_i,
	\] onde\[
		w_i(x) = \psi\left(\frac{|x - x_i|}{\sigma}\right)
	\] e $W(x) = \operatorname{diag}(w_0(x),w_1(x),\dotsc,w_n(x))$.
\end{frame}

\begin{frame}
	A melhor aproximação de $f$ no sentido dos mínimos quadrados ponderados é dada pela função $F^* = \sum_{i=0}^m \alpha_i^*\varphi_i$, em que $(\alpha_0^*,\dotsc,\alpha_m^*)$ é solução do sistema linear $A(x)\alpha^*(x) = b(x)$, onde\[
		A(x) = \begin{bmatrix}
			{\langle\varphi_0,\varphi_0\rangle}_{W(x)} & \cdots & {\langle\varphi_0,\varphi_n\rangle}_{W(x)} \\
			\vdots                                     & \ddots & \vdots                                     \\
			{\langle\varphi_n,\varphi_0\rangle}_{W(x)} & \cdots & {\langle\varphi_n,\varphi_n\rangle}_{W(x)}
		\end{bmatrix}
	\]
	e\[
		b(x) = \begin{bmatrix}
			{\langle f,\varphi_0\rangle}_{W(x)} \\
			\vdots                              \\
			{\langle f,\varphi_n\rangle}_{W(x)}
		\end{bmatrix}.
	\]
\end{frame}

\section{Exemplo de Poisson}


%----------------------------------------------------------------------------------------------------------------------	
\setLayout{mainpoint}
\begin{frame}
	\frametitle{Exemplo de Poisson}
\end{frame}
%----------------------------------------------------------------------------------------------------------------------	



%----------------------------------------------------------------------------------------------------------------------	
\setLayout{vertical}
% \begin{frame}{Título}
% 	...
% \end{frame}
%----------------------------------------------------------------------------------------------------------------------	

\begin{frame}{Equação de Poisson}
	A equação de Poisson é uma EDP da forma\[
		\nabla^2 u(x) = f(x),\quad x\in\Omega,
	\] em que $\nabla^2 u = \operatorname{div}(\operatorname{grad}(u))$.
\end{frame}

\begin{frame}{Condições de fronteira}
	\begin{itemize}
		\item Dirichlet: $u|_{\partial\Omega} = g_D$.
		\item Neumann: $\partial u/\partial n = g_N$ em $\partial\Omega$.
		\item Robin: $\alpha u + \beta\partial u/\partial n = h$ em $\partial\Omega$.
	\end{itemize}
\end{frame}

\begin{frame}{Diferenças finitas}
	Considera-se $\Omega = [a,b]\times[c,d]$. Usando uma malha cartesiana regular com espaçamentos $\delta x$ e $\delta y$, obtém-se\[
		\frac{u_{i+1,j}-2u_{i,j}+u_{i-1,j}}{\delta x^{2}}+\frac{u_{i,j+1}-2u_{i,j}+u_{i,j-1}}{\delta y^{2}}+\mathcal{O}(\delta x^{2}+\delta y^{2})=f_{i,j},
	\]\pause
	ou ainda, desprezando-se os termos de ordem 2,
	\[
		\frac{U_{i+1,j}-2U_{i,j}+U_{i-1,j}}{\delta x^{2}}+\frac{U_{i,j+1}-2U_{i,j}+U_{i,j-1}}{\delta y^{2}}=f_{i,j},
	\]\pause
	onde $U_{i,j}$ representa uma aproximação para a função $u_{i,j}$, avaliada em $(x_{i},y_{j}) = (a+i\delta x, b+j\delta y)$, e $f_{i,j}=f(x_i, y_j)$.
\end{frame}

\end{document}